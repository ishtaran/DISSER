%%% Основные сведения %%%
\newcommand{\thesisAuthor}             % Диссертация, ФИО автора
{%
    \texorpdfstring{% \texorpdfstring takes two arguments and uses the first for (La)TeX and the second for pdf
        Тощев Александр Сергеевич% так будет отображаться на титульном листе или в тексте, где будет использоваться перемная
    }{%
        Тощев, Александр Сергеевич% эта запись для свойств pdf-файла. В таком виде, если pdf будет обработан программами для сбора библиографических сведений, будет правильно представлена фамилия.
    }%
}
\newcommand{\thesisUdk}                % Диссертация, УДК
{004.8}
\newcommand{\thesisTitle}              % Диссертация, название
{\texorpdfstring{\MakeUppercase{Интеллектуальная система повышения эффективности ИТ-службы предприятия}}{Интеллектуальная система повышения эффективности ИТ-службы предприятия}}
\newcommand{\thesisSpecialtyNumber}    % Диссертация, специальность, номер
{\texorpdfstring{{05.13.11}}{05.13.11}}
\newcommand{\thesisSpecialtyTitle}     % Диссертация, специальность, название
{\texorpdfstring{Математическое и программное обеспечение вычислительных машин, комплексов и компьютерных сетей}{Математическое и программное обеспечение вычислительных машин, комплексов и компьютерных сетей}}
\newcommand{\thesisDegree}             % Диссертация, научная степень
{кандидата технических наук}
\newcommand{\thesisCity}               % Диссертация, город защиты
{Казань}
\newcommand{\thesisYear}               % Диссертация, год защиты
{2017}
\newcommand{\thesisOrganization}       % Диссертация, организация
{Казанский (Приволжский) федеральный университет}

\newcommand{\thesisInOrganization}       % Диссертация, организация в предложном падеже: Работа выполнена в ...
{Институте математики и механики (ИММ) им. Н.И. Лобачевского Казанского (Приволжского) федерального университета (КФУ)}

\newcommand{\supervisorFio}            % Научный руководитель, ФИО
{Елизаров Александр Михайлович}
\newcommand{\supervisorRegalia}        % Научный руководитель, регалии
{доктор физико-математических наук, профессор}
\newcommand{\supervisorRegaliaSecond}
{заслуженный деятель науки РТ, \\ зав. кафедрой дифференциальных уравнений \\ Института математики и механики им. Н.И. Лобачевского \\ Казанского (Приволжского) федерального университета}   
\newcommand{\supervisorRegaliaSecondShort}{заслуженный деятель науки РТ,}
\newcommand{\supervisorRegaliaSynopsisSecond} % for synopsis, because we can use acronym early then in main work
{засл. деятель науки РТ, зав. кафедрой дифференциальных уравнений ИММ им. Н.И. Лобачевского КФУ}   
\newcommand{\opponentOneFio}           % Оппонент 1, ФИО
{Райхлин Вадим Абрамович}
\newcommand{\opponentOneRegalia}       % Оппонент 1, регалии
{доктор физико-математических наук, профессор}
\newcommand{\opponentOneJobPlace}      % Оппонент 1, место работы
{Казанский национальный исследовательский технический университет им. А.Н. Туполева (КНИТУ-КАИ)}
\newcommand{\opponentOneJobPost}       % Оппонент 1, должность
{профессор кафедры компьютерных систем}

\newcommand{\opponentTwoFio}           % Оппонент 2, ФИО
{Поляков Владимир Николаевич}
\newcommand{\opponentTwoRegalia}       % Оппонент 2, регалии
{кандидат технических наук, доцент}
\newcommand{\opponentTwoJobPlace}      % Оппонент 2, место работы
{Национальный исследовательский технологический университет МИСиС}
\newcommand{\opponentTwoJobPost}       % Оппонент 2, должность
{доцент кафедры АСУ}

\newcommand{\leadingOrganizationTitle} % Ведущая организация, дополнительные строки
{Федеральный исследовательский центр «Информатика и управление» Российской академии наук (ФИЦ ИУ РАН), г. Москва}

\newcommand{\defenseDate}              % Защита, дата
{25 мая 2017 года в 14:00}
\newcommand{\defenseCouncilNumber}     % Защита, номер диссертационного совета
{Д~212.081.35}
\newcommand{\defenseCouncilTitle}      % Защита, учреждение диссертационного совета
{ФГАОУ ВО Казанский (Приволжский) федеральный университет}
\newcommand{\defenseCouncilAddress}    % Защита, адрес учреждение диссертационного совета%%% Basic information %%%
\newcommand{\thesisAuthor}             % Dissertation, name of author
{%
    \texorpdfstring{% \texorpdfstring takes two arguments and uses the first for (La)TeX and the second for pdf
        Toshchev Alexander Sergeevich 
        % so will appear on the title page or in the text where perenna will be used    
        }{%
        Toshchev Alexander Sergeevich% this entry for properties of a pdf file. In this form, if the pdf is processed by programs to collect bibliographic information, to be correctly represented by the name.
    }%
}
\newcommand{\thesisUdk}                % Dissertation, УДК
{004.8}
\newcommand{\thesisTitle}              % Dissertation, name
{\texorpdfstring{\MakeUppercase{
Intellectual system for increasing the efficiency  of the enterprise IT services}}}
\newcommand{\thesisSpecialtyNumber}    % Dissertation, specialty, number
{\texorpdfstring{{05.13.11}}{05.13.11}}
\newcommand{\thesisSpecialtyTitle}     % Dissertation, specialty, name
{\texorpdfstring{Mathematical and software components of the computers, complexes and computer networks}{Mathematical and software components of the computers, complexes and computer networks}}
\newcommand{\thesisDegree}             % Dissertation, academic degree
{candidate of Engineering Sciences}
\newcommand{\thesisCity}               % Dissertation, city of defense
{Kazan}
\newcommand{\thesisYear}               % Thesis, year of defense
{2017}
\newcommand{\thesisOrganization}       % Dissertation, organization
{Kazan (Volga region) federal university}
\newcommand{\thesisInOrganization}       % Thesis organization in the prepositional case: Work is performed in ...
{The Institute of mathematics and mechanics (IMM) named after N. I. Lobachevsky of Kazan (Volga region) Federal University (КФУ)}

\newcommand{\supervisorFio}            % Scientific supervisor, name
{Elizarov Aleksandr Mikhailovich}
\newcommand{\supervisorRegalia}        % Supervisor, regalia
{Doctor of physico-mathematical Sciences, Professor}
\newcommand{\supervisorRegaliaSecond}
{honored worker of RT science, \\ head of the Department of differential equations \\ Institute of mathematics and mechanics (IMM) named after N. I. Lobachevsky \\ of Kazan (Volga region) Federal University}   
\newcommand{\supervisorRegaliaSecondShort}{honored worker of RT science,}
\newcommand{\supervisorRegaliaSynopsisSecond} % for synopsis, because we can use acronym early then in main work
{honored worker of RT science, head of the Department of differential equations}   
\newcommand{\opponentOneFio}           % Opponent 1, Name
{Raikhlin Vadim Abramovich}
\newcommand{\opponentOneRegalia}       % Opponent 1, regalia
{Doctor of physico-mathematical Sciences, Professor}
\newcommand{\opponentOneJobPlace}      % Opponent 1, place of work
{Kazan national research technical University n. a. A. N. Tupolev (KNITU-KAI)}
\newcommand{\opponentOneJobPost}       % Opponent 1, job
{Professor, Department of computer systems}

\newcommand{\opponentTwoFio}           % Opponent 2, Name
{Polyakov Vladimir Nikolaevich}
\newcommand{\opponentTwoRegalia}       % Opponent 2, regalia
{Candidate of Engineering Sciences, associate Professor}
\newcommand{\opponentTwoJobPlace}      % Opponent 2, place of job
{National research technological University МИСиС}
\newcommand{\opponentTwoJobPost}       % Opponent 2, job
{associate Professor of MIS}

\newcommand{\leadingOrganizationTitle} % A leading organization, additional lines
{Federal research center "Information and control", Russian Academy of Sciences (FIC Yiwu RAS), Moscow}

\newcommand{\defenseDate}              % Defense, date
{May 25th 2017, 14:00}
\newcommand{\defenseCouncilNumber}     % Defense, the number of Dissertation Council
{Д~212.081.35}
\newcommand{\defenseCouncilTitle}      % Defense, the establishment of the dissertation Council
{Federal STATE Autonomous educational institution of Kazan (Volga region) Federal University}
\newcommand{\defenseCouncilAddress}    % Defense, address the establishment of the dissertation Council
{420008, Kazan, Kremlevskaya str., 35}

\newcommand{\defenseSecretaryFio}      % Secretary of the dissertation Council, name
{Enikeev Arslan Ilyasovich}
\newcommand{\defenseSecretaryF}      % Secretary of the dissertation Council, name
{Enikeev}
\newcommand{\defenseSecretaryI}      % Secretary of the dissertation Council, name
{Arslan}
\newcommand{\defenseSecretaryO}      % Secretary of the dissertation Council, name
{Ilyasovich}
\newcommand{\defenseSecretaryRegalia}  % Secretary of the dissertation Council, regalia{candidate of physico-mathematical Sciences, associate Professor}            % For cuts have GOST, for example: GOST Р 7.0.12-2011 + http://base.garant.ru/179724/#block_30000

\newcommand{\synopsisLibrary}          % Auto essay, the name of the library
{Federal STATE Autonomous educational institution "Kazan (Volga region) Federal University» address 420008, Kazan, Kremlevskaya str., 35}
\newcommand{\synopsisDate}             % Auto essay, the date of mailing
{«21»~March~2017}

\newcommand{\keywords}%                 % Key words to metadata of a PDF of the thesis and of the auto essay
{Dissertation, artificial intelligence, model of thinking}


\newcommand{\icl}
{"ICL KME CS" Ltd.}   

\newcommand{\iclshort}
{ICL}

\newcommand{\triplet}
{Critic~--~The selector~--~Way of thinking}   

\newcommand{\tripleta}
{Of Critic~--~Of Selector~--~of way of thinking}   

\newcommand{\etc}
{and so ~on}   

\newcommand{\quoted}[1]{«#1»}

\newcommand{\comma}{,}

\newcommand{\tripletshort}
{${T^3}$}   

{420008, Казань, ул. Кремлевская, 35}

\newcommand{\defenseSecretaryFio}      % Секретарь диссертационного совета, ФИО
{Еникеев Арслан Ильясович}
\newcommand{\defenseSecretaryRegalia}  % Секретарь диссертационного совета, регалии
{канд. физ.-мат. наук, доцент}            % Для сокращений есть ГОСТы, например: ГОСТ Р 7.0.12-2011 + http://base.garant.ru/179724/#block_30000

\newcommand{\synopsisLibrary}          % Автореферат, название библиотеки
{ФГАОУ ВО «Казанский (Приволжский) федеральный университет» по адресу 420008, Казань, ул. Кремлевская, 35}
\newcommand{\synopsisDate}             % Автореферат, дата рассылки
{«~~~»~~~~~~~~~~~~~~~~~~2017 года}

\newcommand{\keywords}%                 % Ключевые слова для метаданных PDF диссертации и автореферата
{диссертация, искусственный интеллект, модели мышления}


\newcommand{\icl}
{ОАО «АйСиЭл КПО-ВС (г. Казань)»}   

\newcommand{\iclshort}
{ICL}

\newcommand{\triplet}
{Критик~--~Селектор~--~Образ мышления}   

\newcommand{\tripleta}
{Критика~--~Селектора~--~Образа мышления}   

\newcommand{\etc}
{и т.~д.}   

\newcommand{\quoted}[1]{«#1»}

\newcommand{\comma}{,}

\newcommand{\tripletshort}
{${T^3}$}   
