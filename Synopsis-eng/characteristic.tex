{\aim} of the thesis is to develop an intelligent system of improving the efficiency of it operations-enterprise services. \par
{\scope}~--- systems development of databases and knowledge.\par
{\subject}  is the registration process and resolving of problem situations arising in infrastructure of the enterprise.\par
{\methods}~--- theoretical methods: simulation, theory of knowledge bases in the field of artificial intelligence; special techniques: system modeling; experimental methods: the method of observation, experimentation.\par 
To achieve this goal we solved the following problems and {\tasks}:
\begin{enumerate}
  \item To analyze the system of knowledge management in supporting the information infrastructure of the enterprise;
  \item To design and build a model of the problem-oriented system of knowledge management for decision-making and optimization of the processes of registration, analysis and processing of user requests in the service area of the information infrastructure of the enterprise;
  \item On the basis of built model to develop the architecture and prototype of an intelligent system of improving the efficiency of it operations-enterprise services;
  \item To hold the testing of the prototype on test data.
\end{enumerate}

\defpositions
\begin{enumerate}
  \item The analysis results of the database management systems knowledge of the it support infrastructure;
  \item The constructed model of the problem-oriented system of knowledge management and optimization of processes of user requests in the service area of enterprise's infrastructure;
  \item Created a prototype software implementation of the problem-oriented model system of knowledge management and the optimization processing of user requests in the service areas of it infrastructure enterprise;
  \item The results of testing of a prototype problem-oriented control system in the control examples.
\end{enumerate}

\novelty\ the conducted research are the following:
\begin{enumerate}
  \item On the basis of generalization of the model of thinking developed by M. Minsky created a simulation model of the problem-oriented system of management, decision-making in the field of maintenance of it infrastructure of enterprise;
  \item Investigated the possibilities of using models of thinking in relation to the service area of the information infrastructure of the enterprise;
  \item The new schema, and the original storage method for a model of thinking that is more efficient than standard ways storage (such as a relational database) was shown;
  \item On the basis of generalization of the Minsky thinking model was created the architecture service system of the information infrastructure of the enterprise and a software prototype of this system.
\end{enumerate}

\influence\ The system developed in the framework of the thesis, has considerable practical. The idea of work affected by the production problems in the it industry, with which the author faced daily in the process of resolving various incidents arising from the activities of technical support \icl~--- one of the largest backbone enterprises of the it industry of the Tatarstan Republic. Therefore it was necessary to develop a deep understanding of a particular subject area, to select an acceptable software solution, to receive a practical application in the organization of informational support of it-infrastructure specific enterprise. \par
\reliability\ scientific results and developed practical recommendations are based on correct formulation and General private the considered problems, using well-known fundamental theoretical provisions, sufficient volume of data used in statistical modelling, and a wide experimental material used for numerical estimates of achievable quality indicators. \par 
Research conducted in the thesis correspond to the passport of the specialty 05.13.11~--- Mathematical and software of computers machines, complexes and computer networks, the mapping is shown in the table \ref{ResearchDescription}.
\renewcommand\tablename{Table} 
\begin{longtable}{|p{8cm}|p{8cm}|}
 \caption[A comparison of the areas of research stipulated in the specialty 05.13.11, and the results obtained in the thesis]{Mapping directions research provided by specialty 05.13.11, and the results obtained in the thesis}\label{ResearchDescription} \\ 
 \hline
 
 \multicolumn{1}{|c|}{\textbf{The research direction}} & \multicolumn{1}{c|}{\textbf{The result of work}}  \\ \hline 
\endfirsthead
\multicolumn{2}{c}%
{{\bfseries \tablename\ \thetable{} -- continued}} \\
\hline \multicolumn{1}{|c|}{\textbf{The research direction}} &
\multicolumn{1}{c|}{\textbf{The result of work}}  \\ \hline 
\endhead
\endfoot

\hline \hline
\endlastfoot
\hline
   Programming languages and system programming, semantics of programs & Developed a semantic model of the knowledge storage organization \\
   \hline
  The database management system and knowledge & Developed a prototype Thinking Understanding (TU) storage systems knowledge and decision-making in the sphere of support of the it infrastructure of the enterprise, which was tested on synthetic data\\
   \hline
   Models and methods of designing programs and software systems for parallel and distributed data processing languages and tools 
for parallel programming & The developed method of parallel processing of expert information and the opportunity for learning with the help of the prototype TU \\
  \end{longtable}


\probation\
 The main results of the thesis were presented at the following conferences:
\begin{itemize}
	\item Tenth 

youth scientific school-conference \quoted{Lobachevskii reading~---2011}. Kazan, October 31st,~--~November 4th 2011;
	\item  International conference"3rd World Conference on Information Technology (WCIT-2012)". Barcelona, 14~--~16 November 2012, Spain; 
	\item II International conference «Artificial intelligence and natural language (AINL-2013)». Saint Petersburg, May 17~--~18 2013 ;
	\item VI International scientific-practical 

conference «E-Kazan 2014». Kazan, April 22~--~24 2014;
	\item XVI All-Russian scientific conference "Digital libraries: 

advanced methods and technologies, digital collections (RCDL-2014)». Dubna, October 13~--~16 2014; 
	\item Seminars on software engineering "All-Kazan Software Engineering Seminar (AKSES-2015)". Kazan, 9 April 2015;
	\item International conference "Agents and multi-agent systems: technologies and applications (AMSTA-2015)". Sorento, 17~--~19 June 2015, Italy.
\end{itemize}Practical approbation of results was carried out during unloading of the incidents of registration system queries, technical support of it infrastructurestructure \icl. The expected was the result in 51\% (the percentage of resolved issues, set by the user), but the result achieved in 30\% we believe acceptable, as it significantly increases the effective resolution problem queries. \par
\contribution\ The author conducted the analysis of user requests and classified them; built a model of the target region and identified opportunities to optimize processes. Data for the study (discharge from registration systems of user requests \iclshort) have been obtained the help of A. V. Krechov.  Together with M. O. by Talanov the author has created a basic architecture of the system. The author has developed system components tested systems on experimental data and fine-tune its work. \par
\publications\ The main results on the topic of the thesis is presented in 10 publications \cite{Lobachevskii, WCIT-2012,  ISGZ, IJSE-1, IJSE-2, RCDL-2014, AMSTA-2015, VAK-1, EB-1, EB-2}, of which articles \cite{RCDL-2014, AMSTA-2015} indexed in Scopus and included in the list of journals of higher attestation Commission of the Russian Federation, article \cite{AMSTA-2015} also indexed in the database Web of Science, work \cite{VAK-1} published in the journal of the higher attestation Commission of the Russian Federation, article \cite{ISGZ} indexed in the database of Russian science citation index, works \cite{Lobachevskii, WCIT-2012, ISGZ} published in the proceedings of international and national conferences, articles \cite{IJSE-1, IJSE-2} published in the international journal "International Journal of Synthetic Emotions"\,, which is included in Association for Computing Machinery. \par
In work \cite{Lobachevskii} A. S. Toshev suggested the original idea of the automatic construction of applications. The article \cite{WCIT-2012} A. S. Toshev developed a software complex, M. O. Talanov suggested the idea, and V. A. Krekhov provided test data from the registration system requests the technical support of it infrastructure \icl. In work \cite{ISGZ} A. S. Toshev proposed and implemented the architecture of an intelligent agent, M. O. Talanov set the task of checking the results of implementation of this approach. In articles \cite{IJSE-1, IJSE-2} A. S. Toshev performed the validation of the model proposed by M. O. Talanov. In the work \cite{RCDL-2014} A. S. Toshev implemented the model. In the article \cite{ AMSTA-2015} A. S. Toshev completed the refinement of the model of thinking, M. O. Talanov set the task of giving versatility system. In the article \cite{VAK-1} A. S. Toshev analyzed the results of the system of registration of requests of technical support of it infrastructure \icl\ and hypothesized about the possibility of automating the permissions part of the query. In the works \cite{EB-1, EB-2} A. S. Toshev conducted development and testing model, M. O. Talanov has developed a basic conceptual idea. \par


