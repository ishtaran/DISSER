\chapter*{Введение}							% Заголовок
\addcontentsline{toc}{chapter}{Введение}	% Добавляем его в оглавление

В настоящее время все более популярным и распространенным становится процесс передачи функций поддержки информационной инфраструктуры (далее~--- ИТ-инфраструктуры) предприятия какой-либо внешней компании (см., например, \cite{StartToOutsource}). Это явление стало называться «ИТ-аутсорсинг» (от анг. ”out source”~--– вне источника). С развитием рынка информационных систем компаниям становится невыгодно держать свой штат службы поддержки, и они отдают эти функции сторонней компании (см. \cite{OutsourceEff}). В некоторых случаях передаются все функции поддержки пользователей: будь-то заявка на ремонт компьютера или же информационный запрос, возникающий из-за простого незнания внутренних процессов компании. В результате создается единая точка входа для пользователей, поддерживаемая сторонней компанией \cite{OutsourceSD}. Обобщая, можно сказать, что на аутсорсинг передают все, что возможно: управление персоналом, уборку помещений, обеспечение питанием, разработку программного обеспечения (далее~--- ПО) (см., например, \cite{OutsourceSoft}) \etc \par
В некоторых областях, например, в области информационных технологий (ИТ) за счет аутсорсинга экономия средств предприятия достигает 30\% (по данным Gartner \cite{OutsourceIT}).
Из-за возросшей популярности бизнеса по аутсорсингу именно в ИТ-области и появления большого количества компаний возникла сильная конкуренция \cite{AUTOS-1}, что привело к снижению цен на услуги и потребовало сокращения издержек компаний. Для поиска путей оптимизации издержек было необходимо применение методов системного анализа для решения сложившихся проблем \cite{AUTOM-1}. Также было отмечено падение рентабельности бизнеса как минимум для малых компаний \cite{OUTSOURCE-RENT}, \cite{OutsourceEff}. В контексте оптимизации издержек в настоящей диссертации рассматриваются модель области, модель системы и ее реализация, которая повышает эффективность работы специалиста технической поддержки (далее~--- специалист) путем частичной (в некоторых случаях, полной) автоматизации обработки инцидентов (случаев, происшествий)  \cite{SDAUTOM}, начиная с разбора запросов, сформулированных на естественном языке, и заканчивая применением найденного решения. \par
Главным требованием к системе повышения эффективности ИТ-службы предприятия является замена части функций, которые сейчас выполняют специалисты:
\begin{enumerate}
  \item Обработка запросов на естественном языке~--- эта функция широко востребована и в системах анализа проблем пользователя с построением статистики «Удовлетворенность пользователя программным продуктом» \cite{TUTUB-1}. Общее понимание проблемы зависит от понимания языка, на котором общаются специалисты;
  \item Возможность обучения. Такая возможность системы позволяет упростить ее эксплуатацию и расширение. По данным исследования \cite{LEARN-1}, возможность обучения очень важна для любой интеллектуальной системы, включая системы управления роботами. Обучение обеспечивает системе большие гибкость и универсальность;
  \item Общение со специалистом. Поддержание диалога (коммуникации)~--- необходимое условие для обучения. Кроме того, социальная функция~--- неотъемлемая часть интеллектуальных систем (см., например, \cite{LEARN-2});
  \item Проведение логических рассуждений (возможность размышлять): аналогия, дедукция, индукция~--- умение обобщить решение одной проблемы и, экстраполируя его, применить для решения других. Иными словами, это возможность для системы принять правильное решение. Например, принятие решений широко используется в интеллектуальных системах управления производством  \cite{LEARN-3}.
\end{enumerate} \par

Интерес к области интеллектуальных систем обработки информации можно, в частности, оценить как количество публикаций за последние годы, процитированных в базе данных Scopus,~--- с 2004 года в среднем оно составило около 1010 в год. \par


На данный момент времени многие компании ведут в различных областях разработку подобных систем, обладающих свойствами, описанными выше. Системы такого класса также называются \textit{вопросно-ответными}. Примером является набирающая популярность IBM Watson \cite{WATSON-PO}, \cite{WATSON-PTOP} (которая является коммерческой и закрытой, информации о ее внутреннем устройстве мало). Другой пример~--- компания HP использует результаты исследования \cite{TUTUB-2} для автоматического определения проблем и степени удовлетворенности пользователей из отчетов об использовании программного обеспечения. Также эта компания работает над автоматическим решением проблем (как описано выше). \par

В настоящей диссертации представлены результаты и апробации создания вопросно-ответной системы на основе исследования целевой области (удаленная поддержка информационной инфраструктуры предприятия) и построения модели системы. Акцент был сделан на создании интеллектуальной системы для решения широкого круга проблем. \par

Следует отметить, что большинство проблем, которые решает удаленная служба поддержки информационной инфраструктуры предприятия, носит достаточно тривиальный характер (по данным компании \icl): установить приложение; переустановить приложение; решить проблему с доступом к тому или иному ресурсу.
Названные проблемы решают специалисты технической поддержки, которая обычно делится на несколько линий по уровню умения специалистов. Каждая линия поддержки представлена своим классом специалистов. В среднем команда, обслуживающая одного заказчика, насчитывает около 60 человек. Как показывают исследования, решение части задач может быть автоматизировано. Если это будет сделано,  специалисты получат дополнительное время для решения более сложных задач. \par


\textbf{Общая характеристика диссертации} 
\newcommand{\actuality}{\underline{\textbf{Актуальность темы.}}}
\newcommand{\aim}{{\textbf{Целью}}}
\newcommand{\tasks}{{\textbf{задачи}}}
\newcommand{\scope}{{\textbf{Область исследования}}}
\newcommand{\subject}{{\textbf{Предметом исследования}}}
\newcommand{\methods}{{\textbf{Методы исследования}}}
\newcommand{\defpositions}{{\textbf{Основные положения, выносимые на~защиту:}}}
\newcommand{\novelty}{{\textbf{Научная новизна}}}
\newcommand{\influence}{{\textbf{Практическая значимость.}}}
\newcommand{\reliability}{{\textbf{Достоверность}}}
\newcommand{\probation}{{\textbf{Апробация работы.}}}
\newcommand{\contribution}{{\textbf{Личный вклад.}}}
\newcommand{\publications}{{\textbf{Публикации.}}}

{\aim} диссертации является разработка интеллектуальной системы повышения эффективности деятельности ИТ-службы предприятия. \par
{\scope}~--- разработка cистем управления базами данных и знаний.\par
{\subject}  является процесс регистрации и устранения проблемных ситуаций, возникающих в ИТ-инфраструктуре предприятия.\par
{\methods}~--- теоретические методы: имитационное моделирование, теория баз знаний в области искусственного интеллекта; специальные методы: системное моделирование; экспериментальные методы: метод наблюдений, проведение экспериментов.\par 
Для достижения поставленной цели были решены следующие проблемы и {\tasks}:
\begin{enumerate}
  \item Провести анализ систем управления базами знаний в области поддержки информационной инфраструктуры предприятия;
  \item Разработать и построить модель проблемно-ориентированной системы управления базой знаний для принятия решений и оптимизации процессов регистрации, анализа и обработки запросов пользователей в области обслуживания информационной инфраструктуры предприятия;
  \item На основе построенной модели разработать архитектуру и создать прототип интеллектуальной системы повышения эффективности деятельности ИТ-службы предприятия;
  \item Провести апробацию прототипа на тестовых данных.
\end{enumerate}

\defpositions
\begin{enumerate}
  \item Результаты анализа систем управления базами знаний в области поддержки ИТ-инфраструктуры предприятия;
  \item Построенная модель проблемно-ориентированной системы управления базой знаний и оптимизации процессов обработки запросов пользователей в области обслуживания ИТ-инфраструктуры предприятия;
  \item Созданный прототип программной реализации модели проблемно-ориентированной системы управления базой знаний и оптимизации обработки запросов пользователей в области обслуживания ИТ-инфраструктуры предприятия;
  \item Результаты апробации прототипа проблемно-ориентированной системы управления на контрольных примерах.
\end{enumerate}

\novelty\ проведенного исследования состоит в следующем:
\begin{enumerate}
  \item На основе обобщения модели мышления, разработанной М. Мински, создана имитационная модель проблемно-ориентированной системы управления, принятия решений в области обслуживания ИТ-инфраструктуры предприятия;
  \item Исследованы возможности использования моделей мышления применительно к области обслуживания информационной инфраструктуры предприятия;
  \item Представлены новая схема данных и оригинальный способ хранения данных для построенной модели мышления, более эффективный по сравнению со стандартными способами хранения (такими, например, как реляционные базы данных);
  \item На основе построенного обобщения модели мышления Мински созданы архитектура системы обслуживания информационной инфраструктуры предприятия и программный прототип этой системы.
\end{enumerate}

\influence\ 
Система, разработанная в рамках диссертации, имеет значимый практический характер. Идея работы зародилась под влиянием производственных проблем в ИТ-отрасли, с которыми автор сталкивался ежедневно в процессе разрешения различных инцидентов, возникающих в деятельности службы технической поддержки \icl~--- одном из крупнейших системообразующих предприятий ИТ-отрасли Республики Татарстан. Поэтому было необходимо выработать глубокое понимание конкретной предметной области, чтобы выбрать приемлемое программное решение, получившее практическое применение при организации информационной поддержки ИТ-инфраструктуры конкретного предприятия. \par
\reliability\ полученных научных результатов и выработанных практических рекомендаций базируется на корректной постановке общих и частных рассматриваемых задач,  использовании известных фундаментальных теоретических положений, достаточном объёме данных, использованных при статистическом моделировании, и широком экспериментальном материале, использованном для численных оценок достижимых качественных показателей. \par 
Исследования, проведенные в диссертации, соответствуют паспорту специальности 05.13.11~--- Математическое и программное обеспечение вычислительных машин, комплексов и компьютерных сетей, сопоставление приведено в таблице \ref{ResearchDescription}.

\begin{longtable}{|p{8cm}|p{8cm}|}
 \caption[Сопоставление направлений исследований, предусмотренных специальностью 05.13.11, и результатов, полученных в диссертации]{Сопоставление направлений исследований предусмотренных специальностью 05.13.11, и результатов, полученных в диссертации}\label{ResearchDescription} \\ 
 \hline
 
 \multicolumn{1}{|c|}{\textbf{Направление исследования}} & \multicolumn{1}{c|}{\textbf{Результат работы}}  \\ \hline 
\endfirsthead
\multicolumn{2}{c}%
{{\bfseries \tablename\ \thetable{} -- продолжение}} \\
\hline \multicolumn{1}{|c|}{\textbf{Направление исследования}} &
\multicolumn{1}{c|}{\textbf{Результат работы}}  \\ \hline 
\endhead
\endfoot

\hline \hline
\endlastfoot
\hline
   Языки программирования и системы программирования, семантика программ & Разработана семантическая модель организации хранения знаний \\
   \hline
  Системы управления базами данных и знаний & Разработан прототип Thinking Understanding (TU) системы хранения знаний и принятия решений в сфере поддержки ИТ-инфраструктуры предприятия, который был испытан на модельных данных\\
   \hline
   Модели и методы создания программ и программных систем для параллельной и распределенной обработки данных, языки и инструментальные средства параллельного программирования & Разработан метод параллельной обработки экспертной информации c возможностью обучения при помощи прототипа TU \\
  \end{longtable}


\probation\
 Основные результаты диссертационной работы докладывались на следующих конференциях:
\begin{itemize}
	\item Десятая молодежная научная школа-конференция \quoted{Лобачевские чтения~---2011}. Казань, 31 октября~--~4 ноября 2011 года;
	\item Международная конференция "3rd World Conference on Information Technology (WCIT-2012)". Barcelona, 14~--~16 November 2012, Spain; 
	\item II Международная конференция «Искусственный интеллект и естественный язык (AINL-2013)». Санкт-Петербург, 17~--~18 мая 2013 года;
	\item VI Международная научно-практическая конференция «Электронная Казань 2014». Казань, 22~--~24 апреля 2014 года;
	\item XVI Всероссийская научная конференция «Электронные библиотеки: перспективные методы и технологии, электронные коллекции (RCDL-2014)». Дубна, 13~--~16 октября 2014 года; 
	\item Семинары по программной инженерии "All-Kazan Software Engineering Seminar (AKSES-2015)". Kazan, 9 April 2015;
	\item Международная конференция "Agents and multi-agent systems: technologies and applications (AMSTA-2015)". Sorento, 17~--~19 June 2015, Italy.
\end{itemize} \par
Практическая апробация результатов работы проводилась на выгрузке инцидентов из системы регистрации запросов службы технической поддержки ИТ-инфраструктуры \icl. Ожидаемым был результат в 51\% (доля разрешенных проблем, поставленных пользователями), но и достигнутый результат в 30\% мы считаем приемлемым, так как он значительно увеличивает эффективность разрешения проблемных запросов. \par
\contribution\ Автор провел анализ запросов пользователей и классифицировал их; построил модель целевой области и выявил возможности оптимизации процессов в ней. Данные для исследования (выгрузка из систем регистрации запросов пользователей \iclshort) были получены при помощи А.В. Крехова.  Совместно с М.О. Талановым автор создал базовую архитектуру системы. Автор разработал компоненты системы, провел испытание системы на экспериментальных данных и отладил ее работу. \par
\publications\ Основные результаты по теме диссертации изложены в 10 печатных изданиях  \cite{Lobachevskii, WCIT-2012,  ISGZ, IJSE-1, IJSE-2, RCDL-2014, AMSTA-2015, VAK-1, EB-1, EB-2}, из которых статьи \cite{RCDL-2014, AMSTA-2015} проиндексированы в БД Scopus и входят в перечень журналов ВАК РФ, статья \cite{AMSTA-2015} проиндексирована также в БД Web of Science, работа \cite{VAK-1} опубликована в журнале из перечня ВАК РФ, статья \cite{ISGZ} проиндексирована в БД РИНЦ, работы \cite{Lobachevskii, WCIT-2012, ISGZ} опубликованы в материалах международных и всероссийских конференций, статьи \cite{IJSE-1, IJSE-2} опубликованы в международном журнале "International Journal of Synthetic Emotions"\,, входящем в индекс ACM (Association for Computing Machinery). \par
В работе  \cite{Lobachevskii} А.С. Тощев предложил оригинальную идею автоматического конструирования приложений. В статье \cite{WCIT-2012} А.С. Тощевым был разработан программный комплекс, М.О. Таланов предложил идею, а А.В. Крехов предоставил тестовые данные из системы регистрации запросов службы технической поддержки ИТ-инфраструктуры \icl. В работе \cite{ISGZ} А.С. Тощев предложил и реализовал архитектуру интеллектуального агента, М.О. Таланов поставил задачу проверки результатов реализации этого подхода. В статьях \cite{IJSE-1, IJSE-2} А.С. Тощев выполнил проверку модели, предложенной М.О. Талановым. В работе \cite{RCDL-2014} А.С. Тощев реализовал модель. В статье \cite{ AMSTA-2015} А.С. Тощев выполнил доработку модели мышления, М.О. Таланов поставил задачу придания универсальности системе. В статье \cite{VAK-1} А.С. Тощев проанализировал результаты работы системы регистрации запросов службы технической поддержки ИТ-инфраструктуры \icl\ и выдвинул гипотезу о возможности автоматизации разрешения части запросов. В работах \cite{EB-1, EB-2} А.С. Тощев провел разработку и проверку модели, М.О. Таланов разработал основную концептуальную идею. \par


 % Характеристика работы по структуре во введении и в автореферате не отличается (ГОСТ Р 7.0.11, пункты 5.3.1 и 9.2.1), потому её загружаем из одного и того же внешнего файла, предварительно задав форму выделения некоторым параметрам
%% регистрируем счётчики в системе totcounter
\regtotcounter{totalcount@figure}
\regtotcounter{totalcount@table}       % Если поставить в преамбуле то ошибка в числе таблиц
\regtotcounter{TotPages}               % Если поставить в преамбуле то ошибка в числе страниц

\textbf{Объем и структура работы.} Диссертация состоит из введения, четырех глав, заключения и пяти приложений. Полный объём диссертации составляет \formbytotal{TotPages}{страниц}{у}{ы}{} 
с~\formbytotal{totalcount@figure}{рисунк}{ом}{ами}{ами}
и~\formbytotal{totalcount@table}{таблиц}{ей}{ами}{ами}. Список литературы содержит  
\formbytotal{citenum}{наименован}{ие}{ия}{ий}.
\clearpage

