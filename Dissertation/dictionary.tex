\chapter*{Словарь терминов} \label{Glossary}            % Заголовок
\addcontentsline{toc}{chapter}{Словарь терминов}  % Добавляем его в оглавление

\textbf{База Знаний}~--- База данных приложения, представленная в виде онтологии знаний. \par
\textbf{WayToThink}~--- Путь мышления. Основан на определении Марвина Мински \cite{EmotionMachine}. Класс объектов, которые модифицируют данные. \par
\textbf{Critic}~--- Основан на определении Марвина Мински \cite{EmotionMachine}. Класс объектов, которые выступают триггерами при наступление определенного события. \par
\textbf{ThinkingLifeCycle}~--- Основан на определении Марвина Мински \cite{EmotionMachine}. Класс объектов, которые выступают основными объектами для запуска в приложении~--- рабочими процессами. \par
\textbf{Selector}~--- Компонент, отвечающий за выборку данных из Базы Знаний. \par
\textbf{Instinctive}~--- Инстинктивный уровень. \par
\textbf{Learned}~--- Уровень обученных реакций. \par
\textbf{Deliberative}~--- Уровень рассуждений. \par
\textbf{Reflective}~--- Рефлексивный уровень. \par
\textbf{Self-Reflective Thinking	}~--- Саморефлексивный уровень. \par
\textbf{Self-Conscious Reflection}~--- Самосознательный уровень. \par
\textbf{ThinkingUnderstanding}~--- Система, созданная в рамках работы. Дословный перевод «Мышление-Понимание».\par  
\textbf{Вариант использования}~--- Термин из стандарта UML, который описывает возможные способы функционирования системы.\par  
\textbf{Диаграмма действий}~--- Термин из стандарта UML, который описывает последовательность действий пользователя.\par  
 
\clearpage