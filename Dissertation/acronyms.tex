\chapter*{Список сокращений и условных обозначений} \label{acronyms}             % Заголовок
\addcontentsline{toc}{chapter}{Список сокращений и условных обозначений}  % Добавляем его в оглавление

\textbf{selectLinkedObject(obj:Resource, linkName:String): Link<Resource>}~--- Описание метода. selectLinkedObject~--- название метода. (obj:Resource, linkName:String)~--- параметры метода. linkName~--- имя параметра. String тип данных. Link<Resource>~--- тип возвращаемых данных. Если метод данных не возвращает, то ничего не указывается.\par
\textbf{DomainModel:SemanticNetwork}~--- Описание класса, где DomainModel~--- сам класс, а SemanticNetwork~--- класс-родитель.\par
\textbf{TU}~--- Сокращение от ThinkingUnderstanding.\par
\textbf{TLC}~--- Thinking Life Cycle.\par
\textbf{НДФЛ}~--- Налог на доходы физически лиц.\par
\textbf{ПО}~--- Программное обеспечение.\par
\textbf{ФБ}~--- Федеральный бюджет.\par
\textbf{ПФР}~--- Пенсионный фонд России.\par
\textbf{ТФОМС}~--- Территориальный фонд обязательного медицинского страхования.\par
\textbf{ФФОМС}~--- Федеральный фонд обязательного медицинского страхования.\par
\textbf{ФСС}~--- Фонд социального страхования.\par
\textbf{БД}~--- База данных.\par
\textbf{мс.}~--- Миллисекунды.\par


\clearpage
